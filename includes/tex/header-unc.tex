
%Font packages
%\usepackage[T1]{fontenc}
%\usepackage[utf8]{inputenc}
\usepackage{csquotes}
\usepackage[english]{babel}
\usepackage[utf8]{inputenc}

\DeclareUnicodeCharacter{2212}{-}

% List of acronyms
\usepackage{longtable}
%\usepackage[acronym]{glossaries}%NOTE: this will NOT work in markdown to latex. have to have access to the latex file with same name to access the glossary.

\usepackage{lscape}
\usepackage{tipa}

%% Extra packages that I added -----------------
\usepackage{textgreek}
\usepackage[table]{xcolor}
\usepackage{fancyhdr}
\usepackage{booktabs}
\usepackage{setspace}
\usepackage{kvoptions}
%\usepackage{afterpage}
\usepackage{rotating}%for sidewaystable in xtable
%\usepackage{hyperref}% to highlight links in different colors
%\hypersetup{
%    colorlinks=true,
%    linkcolor=blue,
%    filecolor=magenta,      
%    urlcolor=cyan,
%}

% see the following link for info on biblatex sort order issue: 
% http://tex.stackexchange.com/questions/51434/biblatex-citation-order
\usepackage[style=numeric,
%        hyperref=true,
        maxbibnames=99,
        firstinits=true,
        uniquename=init,
        doi=true,
%        backref=true,
        backend=biber]{biblatex}

\usepackage{float}% for placement of figures and tables

\usepackage{tikz}%for DAG
\usetikzlibrary{arrows.meta,positioning}

\usepackage{array}%for smaller landscape tables
\usepackage{graphicx}

\newcolumntype{Z}[1]{>{\raggedright\let\newline\\\arraybackslash\hspace{0pt}}m{#1}}
% \newcolumntype{C}[1]{>{\centering\let\newline\\\arraybackslash\hspace{0pt}}m{#1}}
\newcolumntype{R}[1]{>{\raggedleft\let\newline\\\arraybackslash\hspace{0pt}}m{#1}}

\usepackage{tabulary}

\usepackage[bf,singlelinecheck=off]{caption}
\usepackage{eso-pic,graphicx,transparent}% see http://stackoverflow.com/questions/32748248/watermark-in-rmarkdown

% see https://tex.stackexchange.com/questions/258688/how-to-remove-double-spacing-in-table-cell-wrap
% \usepackage{etoolbox}
% \BeforeBeginEnvironment{tabular}{\begin{singlespace}}
% \AfterEndEnvironment{tabular}{\end{singlespace}}


%%%%%%%%%%%%%%%%%%%%%%%%%%%%%%%%%%%%%%%%%%%%%%%%%%%%%%%%%%%%%
% GLOSSARIES AND ABBREVIATIONS
%%%%%%%%%%%%%%%%%%%%%%%%%%%%%%%%%%%%%%%%%%%%%%%%%%%%%%%%%%%%%
% To update the printed glossary, you need to run:
% - pdflatex dissertation
% - makeglossaries dissertation
% - pdflatex dissertation
% On Windows, you might need to install Perl first.

%% MY own note: Since I am doing this in Rmd and not LaTex need to make a separate
%% LaTex file, glossary.tex, and add the acronyms there.

%\newacronym{unc}{UNC}{The University of North Carolina at Chapel Hill}
%\makeglossaries

% NOTE: the glossaries package in LaTex does not work when using markdown. Trying another approach from "Using LaTex to Write a PhD Thesis" by Nicola L.C. Talbot



\usepackage{datagidx}% NOTE: this will not work unless you disable glossaries package

\newgidx{acronym}{{\vspace*{1\baselineskip} \normalfont\bfseries{LIST OF ABBREVIATIONS}}} % 2 line spaced after title
%\newgidx{acronym}{\large{LIST OF ABBREVIATIONS}} % 2 line spaced after title
%\DTLgidxCategorySep{1cm}

\DTLgidxSetDefaultDB{acronym}
\newacro{TC}{Total Cholesterol}
\newacro{GRS}{Genetic Risk Score}
 % has acronyms
%\input{includes/tex/glossaryfile-math.tex} % has glossaries

%\usepackage{datagidx}% NOTE: this will not work unless you disable glossaries package
%  \newgidx{glossary}{Glossary}
%  \newgidx{acronym}{List of Abbreviations}
%  
%  \DTLgidxSetDefaultDB{acronym}
%  \newacro{TC}{Total Cholesterol}
%no need to call makeglossaries.


% \newgidx{glossary}{Glossary}
% \newgidx{acronym}{List of Abbreviations}


% \DTLgidxSetDefaultDB{glossary}
% \newterm
% [%
% description={Total cholesterol},% brief description
% ]%
% {tc}% the name



%% Math Packages %%%%%%%%%%%%%%%%%%%%%%%%%%%%%%%%%%%%%%%%%%%%
\usepackage{amsmath}
\usepackage{amsthm}
\usepackage{amsfonts}
\usepackage{bbm}
\usepackage{amssymb}
\usepackage{geometry}

%% NOTE: I commented this section out. Not sure why but it kept my markdown file from compiling, kicking up a 'error 43'. Do I need to put this back? not sure.
%% Reduce spacing between paragraph and section title %%%%%%%
%% @todo: Put this modification in the class file itself.
% \usepackage{titlesec}
% \titlespacing*{\section}
% {0pt}{-5pt}{0pt}
% \titlespacing*{\subsection}
% {0pt}{-5pt}{0pt}
\usepackage{indentfirst}   %Indents first paragraphs in every section.

\setlength\parindent{24pt}

%% Flush footnotes to the left
\usepackage[hang,flushmargin]{footmisc}
%% Places footnotes immediately below horizontal rule
\setlength{\footnotesep}{0pt}

%% Normal LaTeX or pdfLaTeX? %%%%%%%%%%%%%%%%%%%%%%%%%%%%%%%%
\RequirePackage{ifpdf}

% %% Packages for Graphics & Figures %%%%%%%%%%%%%%%%%%%%%%%%%%
% \ifpdf %%Inclusion of graphics via \includegraphics{file}
% 	\usepackage[pdftex]{graphicx} %%graphics in pdfLaTeX
% \else
% 	\usepackage[dvips]{graphicx} %%graphics and normal LaTeX
% \fi


\usepackage{adjustbox}

%% EXTRA
% %%%%%%%%%%%%%%%%%%%%%%%%%%%%%%%%%%%%%%%%%%%%%%%%%%%%%%%%%%%%%%%
\usepackage{pdflscape}
\newcommand{\blandscape}{\begin{landscape}}
\newcommand{\elandscape}{\end{landscape}}

\colorlet{vlg}{gray!20}

\usepackage{tocloft}

% see https://tex.stackexchange.com/questions/4152/how-do-i-prevent-widow-orphan-lines
%\usepackage[all]{nowidow}
% Disallow all widows and orphans (clubs)
% \widowpenalty=10000
% \clubpenalty=10000

